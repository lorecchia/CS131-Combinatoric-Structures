
% Default to the notebook output style

    


% Inherit from the specified cell style.




    
\documentclass[11pt]{article}

    
    
    \usepackage[T1]{fontenc}
    % Nicer default font than Computer Modern for most use cases
    \usepackage{palatino}

    % Basic figure setup, for now with no caption control since it's done
    % automatically by Pandoc (which extracts ![](path) syntax from Markdown).
    \usepackage{graphicx}
    % We will generate all images so they have a width \maxwidth. This means
    % that they will get their normal width if they fit onto the page, but
    % are scaled down if they would overflow the margins.
    \makeatletter
    \def\maxwidth{\ifdim\Gin@nat@width>\linewidth\linewidth
    \else\Gin@nat@width\fi}
    \makeatother
    \let\Oldincludegraphics\includegraphics
    % Set max figure width to be 80% of text width, for now hardcoded.
    \renewcommand{\includegraphics}[1]{\Oldincludegraphics[width=.8\maxwidth]{#1}}
    % Ensure that by default, figures have no caption (until we provide a
    % proper Figure object with a Caption API and a way to capture that
    % in the conversion process - todo).
    \usepackage{caption}
    \DeclareCaptionLabelFormat{nolabel}{}
    \captionsetup{labelformat=nolabel}

    \usepackage{adjustbox} % Used to constrain images to a maximum size 
    \usepackage{xcolor} % Allow colors to be defined
    \usepackage{enumerate} % Needed for markdown enumerations to work
    \usepackage{geometry} % Used to adjust the document margins
    \usepackage{amsmath} % Equations
    \usepackage{amssymb} % Equations
    \usepackage{textcomp} % defines textquotesingle
    % Hack from http://tex.stackexchange.com/a/47451/13684:
    \AtBeginDocument{%
        \def\PYZsq{\textquotesingle}% Upright quotes in Pygmentized code
    }
    \usepackage{upquote} % Upright quotes for verbatim code
    \usepackage{eurosym} % defines \euro
    \usepackage[mathletters]{ucs} % Extended unicode (utf-8) support
    \usepackage[utf8x]{inputenc} % Allow utf-8 characters in the tex document
    \usepackage{fancyvrb} % verbatim replacement that allows latex
    \usepackage{grffile} % extends the file name processing of package graphics 
                         % to support a larger range 
    % The hyperref package gives us a pdf with properly built
    % internal navigation ('pdf bookmarks' for the table of contents,
    % internal cross-reference links, web links for URLs, etc.)
    \usepackage{hyperref}
    \usepackage{longtable} % longtable support required by pandoc >1.10
    \usepackage{booktabs}  % table support for pandoc > 1.12.2
    \usepackage[normalem]{ulem} % ulem is needed to support strikethroughs (\sout)
                                % normalem makes italics be italics, not underlines
    

    
    
    % Colors for the hyperref package
    \definecolor{urlcolor}{rgb}{0,.145,.698}
    \definecolor{linkcolor}{rgb}{.71,0.21,0.01}
    \definecolor{citecolor}{rgb}{.12,.54,.11}

    % ANSI colors
    \definecolor{ansi-black}{HTML}{3E424D}
    \definecolor{ansi-black-intense}{HTML}{282C36}
    \definecolor{ansi-red}{HTML}{E75C58}
    \definecolor{ansi-red-intense}{HTML}{B22B31}
    \definecolor{ansi-green}{HTML}{00A250}
    \definecolor{ansi-green-intense}{HTML}{007427}
    \definecolor{ansi-yellow}{HTML}{DDB62B}
    \definecolor{ansi-yellow-intense}{HTML}{B27D12}
    \definecolor{ansi-blue}{HTML}{208FFB}
    \definecolor{ansi-blue-intense}{HTML}{0065CA}
    \definecolor{ansi-magenta}{HTML}{D160C4}
    \definecolor{ansi-magenta-intense}{HTML}{A03196}
    \definecolor{ansi-cyan}{HTML}{60C6C8}
    \definecolor{ansi-cyan-intense}{HTML}{258F8F}
    \definecolor{ansi-white}{HTML}{C5C1B4}
    \definecolor{ansi-white-intense}{HTML}{A1A6B2}

    % commands and environments needed by pandoc snippets
    % extracted from the output of `pandoc -s`
    \providecommand{\tightlist}{%
      \setlength{\itemsep}{0pt}\setlength{\parskip}{0pt}}
    \DefineVerbatimEnvironment{Highlighting}{Verbatim}{commandchars=\\\{\}}
    % Add ',fontsize=\small' for more characters per line
    \newenvironment{Shaded}{}{}
    \newcommand{\KeywordTok}[1]{\textcolor[rgb]{0.00,0.44,0.13}{\textbf{{#1}}}}
    \newcommand{\DataTypeTok}[1]{\textcolor[rgb]{0.56,0.13,0.00}{{#1}}}
    \newcommand{\DecValTok}[1]{\textcolor[rgb]{0.25,0.63,0.44}{{#1}}}
    \newcommand{\BaseNTok}[1]{\textcolor[rgb]{0.25,0.63,0.44}{{#1}}}
    \newcommand{\FloatTok}[1]{\textcolor[rgb]{0.25,0.63,0.44}{{#1}}}
    \newcommand{\CharTok}[1]{\textcolor[rgb]{0.25,0.44,0.63}{{#1}}}
    \newcommand{\StringTok}[1]{\textcolor[rgb]{0.25,0.44,0.63}{{#1}}}
    \newcommand{\CommentTok}[1]{\textcolor[rgb]{0.38,0.63,0.69}{\textit{{#1}}}}
    \newcommand{\OtherTok}[1]{\textcolor[rgb]{0.00,0.44,0.13}{{#1}}}
    \newcommand{\AlertTok}[1]{\textcolor[rgb]{1.00,0.00,0.00}{\textbf{{#1}}}}
    \newcommand{\FunctionTok}[1]{\textcolor[rgb]{0.02,0.16,0.49}{{#1}}}
    \newcommand{\RegionMarkerTok}[1]{{#1}}
    \newcommand{\ErrorTok}[1]{\textcolor[rgb]{1.00,0.00,0.00}{\textbf{{#1}}}}
    \newcommand{\NormalTok}[1]{{#1}}
    
    % Additional commands for more recent versions of Pandoc
    \newcommand{\ConstantTok}[1]{\textcolor[rgb]{0.53,0.00,0.00}{{#1}}}
    \newcommand{\SpecialCharTok}[1]{\textcolor[rgb]{0.25,0.44,0.63}{{#1}}}
    \newcommand{\VerbatimStringTok}[1]{\textcolor[rgb]{0.25,0.44,0.63}{{#1}}}
    \newcommand{\SpecialStringTok}[1]{\textcolor[rgb]{0.73,0.40,0.53}{{#1}}}
    \newcommand{\ImportTok}[1]{{#1}}
    \newcommand{\DocumentationTok}[1]{\textcolor[rgb]{0.73,0.13,0.13}{\textit{{#1}}}}
    \newcommand{\AnnotationTok}[1]{\textcolor[rgb]{0.38,0.63,0.69}{\textbf{\textit{{#1}}}}}
    \newcommand{\CommentVarTok}[1]{\textcolor[rgb]{0.38,0.63,0.69}{\textbf{\textit{{#1}}}}}
    \newcommand{\VariableTok}[1]{\textcolor[rgb]{0.10,0.09,0.49}{{#1}}}
    \newcommand{\ControlFlowTok}[1]{\textcolor[rgb]{0.00,0.44,0.13}{\textbf{{#1}}}}
    \newcommand{\OperatorTok}[1]{\textcolor[rgb]{0.40,0.40,0.40}{{#1}}}
    \newcommand{\BuiltInTok}[1]{{#1}}
    \newcommand{\ExtensionTok}[1]{{#1}}
    \newcommand{\PreprocessorTok}[1]{\textcolor[rgb]{0.74,0.48,0.00}{{#1}}}
    \newcommand{\AttributeTok}[1]{\textcolor[rgb]{0.49,0.56,0.16}{{#1}}}
    \newcommand{\InformationTok}[1]{\textcolor[rgb]{0.38,0.63,0.69}{\textbf{\textit{{#1}}}}}
    \newcommand{\WarningTok}[1]{\textcolor[rgb]{0.38,0.63,0.69}{\textbf{\textit{{#1}}}}}
    
    
    % Define a nice break command that doesn't care if a line doesn't already
    % exist.
    \def\br{\hspace*{\fill} \\* }
    % Math Jax compatability definitions
    \def\gt{>}
    \def\lt{<}
    % Document parameters
    \title{L04LogicalEquivalences2}
    
    
    

    % Pygments definitions
    
\makeatletter
\def\PY@reset{\let\PY@it=\relax \let\PY@bf=\relax%
    \let\PY@ul=\relax \let\PY@tc=\relax%
    \let\PY@bc=\relax \let\PY@ff=\relax}
\def\PY@tok#1{\csname PY@tok@#1\endcsname}
\def\PY@toks#1+{\ifx\relax#1\empty\else%
    \PY@tok{#1}\expandafter\PY@toks\fi}
\def\PY@do#1{\PY@bc{\PY@tc{\PY@ul{%
    \PY@it{\PY@bf{\PY@ff{#1}}}}}}}
\def\PY#1#2{\PY@reset\PY@toks#1+\relax+\PY@do{#2}}

\expandafter\def\csname PY@tok@vi\endcsname{\def\PY@tc##1{\textcolor[rgb]{0.10,0.09,0.49}{##1}}}
\expandafter\def\csname PY@tok@vc\endcsname{\def\PY@tc##1{\textcolor[rgb]{0.10,0.09,0.49}{##1}}}
\expandafter\def\csname PY@tok@kd\endcsname{\let\PY@bf=\textbf\def\PY@tc##1{\textcolor[rgb]{0.00,0.50,0.00}{##1}}}
\expandafter\def\csname PY@tok@si\endcsname{\let\PY@bf=\textbf\def\PY@tc##1{\textcolor[rgb]{0.73,0.40,0.53}{##1}}}
\expandafter\def\csname PY@tok@w\endcsname{\def\PY@tc##1{\textcolor[rgb]{0.73,0.73,0.73}{##1}}}
\expandafter\def\csname PY@tok@o\endcsname{\def\PY@tc##1{\textcolor[rgb]{0.40,0.40,0.40}{##1}}}
\expandafter\def\csname PY@tok@ch\endcsname{\let\PY@it=\textit\def\PY@tc##1{\textcolor[rgb]{0.25,0.50,0.50}{##1}}}
\expandafter\def\csname PY@tok@mi\endcsname{\def\PY@tc##1{\textcolor[rgb]{0.40,0.40,0.40}{##1}}}
\expandafter\def\csname PY@tok@se\endcsname{\let\PY@bf=\textbf\def\PY@tc##1{\textcolor[rgb]{0.73,0.40,0.13}{##1}}}
\expandafter\def\csname PY@tok@cpf\endcsname{\let\PY@it=\textit\def\PY@tc##1{\textcolor[rgb]{0.25,0.50,0.50}{##1}}}
\expandafter\def\csname PY@tok@c\endcsname{\let\PY@it=\textit\def\PY@tc##1{\textcolor[rgb]{0.25,0.50,0.50}{##1}}}
\expandafter\def\csname PY@tok@m\endcsname{\def\PY@tc##1{\textcolor[rgb]{0.40,0.40,0.40}{##1}}}
\expandafter\def\csname PY@tok@mo\endcsname{\def\PY@tc##1{\textcolor[rgb]{0.40,0.40,0.40}{##1}}}
\expandafter\def\csname PY@tok@kc\endcsname{\let\PY@bf=\textbf\def\PY@tc##1{\textcolor[rgb]{0.00,0.50,0.00}{##1}}}
\expandafter\def\csname PY@tok@gu\endcsname{\let\PY@bf=\textbf\def\PY@tc##1{\textcolor[rgb]{0.50,0.00,0.50}{##1}}}
\expandafter\def\csname PY@tok@ne\endcsname{\let\PY@bf=\textbf\def\PY@tc##1{\textcolor[rgb]{0.82,0.25,0.23}{##1}}}
\expandafter\def\csname PY@tok@nl\endcsname{\def\PY@tc##1{\textcolor[rgb]{0.63,0.63,0.00}{##1}}}
\expandafter\def\csname PY@tok@s2\endcsname{\def\PY@tc##1{\textcolor[rgb]{0.73,0.13,0.13}{##1}}}
\expandafter\def\csname PY@tok@nv\endcsname{\def\PY@tc##1{\textcolor[rgb]{0.10,0.09,0.49}{##1}}}
\expandafter\def\csname PY@tok@na\endcsname{\def\PY@tc##1{\textcolor[rgb]{0.49,0.56,0.16}{##1}}}
\expandafter\def\csname PY@tok@ge\endcsname{\let\PY@it=\textit}
\expandafter\def\csname PY@tok@err\endcsname{\def\PY@bc##1{\setlength{\fboxsep}{0pt}\fcolorbox[rgb]{1.00,0.00,0.00}{1,1,1}{\strut ##1}}}
\expandafter\def\csname PY@tok@nn\endcsname{\let\PY@bf=\textbf\def\PY@tc##1{\textcolor[rgb]{0.00,0.00,1.00}{##1}}}
\expandafter\def\csname PY@tok@go\endcsname{\def\PY@tc##1{\textcolor[rgb]{0.53,0.53,0.53}{##1}}}
\expandafter\def\csname PY@tok@ni\endcsname{\let\PY@bf=\textbf\def\PY@tc##1{\textcolor[rgb]{0.60,0.60,0.60}{##1}}}
\expandafter\def\csname PY@tok@vg\endcsname{\def\PY@tc##1{\textcolor[rgb]{0.10,0.09,0.49}{##1}}}
\expandafter\def\csname PY@tok@gr\endcsname{\def\PY@tc##1{\textcolor[rgb]{1.00,0.00,0.00}{##1}}}
\expandafter\def\csname PY@tok@sx\endcsname{\def\PY@tc##1{\textcolor[rgb]{0.00,0.50,0.00}{##1}}}
\expandafter\def\csname PY@tok@cp\endcsname{\def\PY@tc##1{\textcolor[rgb]{0.74,0.48,0.00}{##1}}}
\expandafter\def\csname PY@tok@nb\endcsname{\def\PY@tc##1{\textcolor[rgb]{0.00,0.50,0.00}{##1}}}
\expandafter\def\csname PY@tok@no\endcsname{\def\PY@tc##1{\textcolor[rgb]{0.53,0.00,0.00}{##1}}}
\expandafter\def\csname PY@tok@ow\endcsname{\let\PY@bf=\textbf\def\PY@tc##1{\textcolor[rgb]{0.67,0.13,1.00}{##1}}}
\expandafter\def\csname PY@tok@nc\endcsname{\let\PY@bf=\textbf\def\PY@tc##1{\textcolor[rgb]{0.00,0.00,1.00}{##1}}}
\expandafter\def\csname PY@tok@gp\endcsname{\let\PY@bf=\textbf\def\PY@tc##1{\textcolor[rgb]{0.00,0.00,0.50}{##1}}}
\expandafter\def\csname PY@tok@gd\endcsname{\def\PY@tc##1{\textcolor[rgb]{0.63,0.00,0.00}{##1}}}
\expandafter\def\csname PY@tok@sb\endcsname{\def\PY@tc##1{\textcolor[rgb]{0.73,0.13,0.13}{##1}}}
\expandafter\def\csname PY@tok@gi\endcsname{\def\PY@tc##1{\textcolor[rgb]{0.00,0.63,0.00}{##1}}}
\expandafter\def\csname PY@tok@ss\endcsname{\def\PY@tc##1{\textcolor[rgb]{0.10,0.09,0.49}{##1}}}
\expandafter\def\csname PY@tok@kp\endcsname{\def\PY@tc##1{\textcolor[rgb]{0.00,0.50,0.00}{##1}}}
\expandafter\def\csname PY@tok@cm\endcsname{\let\PY@it=\textit\def\PY@tc##1{\textcolor[rgb]{0.25,0.50,0.50}{##1}}}
\expandafter\def\csname PY@tok@sr\endcsname{\def\PY@tc##1{\textcolor[rgb]{0.73,0.40,0.53}{##1}}}
\expandafter\def\csname PY@tok@kr\endcsname{\let\PY@bf=\textbf\def\PY@tc##1{\textcolor[rgb]{0.00,0.50,0.00}{##1}}}
\expandafter\def\csname PY@tok@il\endcsname{\def\PY@tc##1{\textcolor[rgb]{0.40,0.40,0.40}{##1}}}
\expandafter\def\csname PY@tok@c1\endcsname{\let\PY@it=\textit\def\PY@tc##1{\textcolor[rgb]{0.25,0.50,0.50}{##1}}}
\expandafter\def\csname PY@tok@nd\endcsname{\def\PY@tc##1{\textcolor[rgb]{0.67,0.13,1.00}{##1}}}
\expandafter\def\csname PY@tok@sh\endcsname{\def\PY@tc##1{\textcolor[rgb]{0.73,0.13,0.13}{##1}}}
\expandafter\def\csname PY@tok@sd\endcsname{\let\PY@it=\textit\def\PY@tc##1{\textcolor[rgb]{0.73,0.13,0.13}{##1}}}
\expandafter\def\csname PY@tok@sc\endcsname{\def\PY@tc##1{\textcolor[rgb]{0.73,0.13,0.13}{##1}}}
\expandafter\def\csname PY@tok@k\endcsname{\let\PY@bf=\textbf\def\PY@tc##1{\textcolor[rgb]{0.00,0.50,0.00}{##1}}}
\expandafter\def\csname PY@tok@gt\endcsname{\def\PY@tc##1{\textcolor[rgb]{0.00,0.27,0.87}{##1}}}
\expandafter\def\csname PY@tok@mf\endcsname{\def\PY@tc##1{\textcolor[rgb]{0.40,0.40,0.40}{##1}}}
\expandafter\def\csname PY@tok@cs\endcsname{\let\PY@it=\textit\def\PY@tc##1{\textcolor[rgb]{0.25,0.50,0.50}{##1}}}
\expandafter\def\csname PY@tok@mh\endcsname{\def\PY@tc##1{\textcolor[rgb]{0.40,0.40,0.40}{##1}}}
\expandafter\def\csname PY@tok@kn\endcsname{\let\PY@bf=\textbf\def\PY@tc##1{\textcolor[rgb]{0.00,0.50,0.00}{##1}}}
\expandafter\def\csname PY@tok@mb\endcsname{\def\PY@tc##1{\textcolor[rgb]{0.40,0.40,0.40}{##1}}}
\expandafter\def\csname PY@tok@gs\endcsname{\let\PY@bf=\textbf}
\expandafter\def\csname PY@tok@kt\endcsname{\def\PY@tc##1{\textcolor[rgb]{0.69,0.00,0.25}{##1}}}
\expandafter\def\csname PY@tok@nf\endcsname{\def\PY@tc##1{\textcolor[rgb]{0.00,0.00,1.00}{##1}}}
\expandafter\def\csname PY@tok@nt\endcsname{\let\PY@bf=\textbf\def\PY@tc##1{\textcolor[rgb]{0.00,0.50,0.00}{##1}}}
\expandafter\def\csname PY@tok@s\endcsname{\def\PY@tc##1{\textcolor[rgb]{0.73,0.13,0.13}{##1}}}
\expandafter\def\csname PY@tok@bp\endcsname{\def\PY@tc##1{\textcolor[rgb]{0.00,0.50,0.00}{##1}}}
\expandafter\def\csname PY@tok@s1\endcsname{\def\PY@tc##1{\textcolor[rgb]{0.73,0.13,0.13}{##1}}}
\expandafter\def\csname PY@tok@gh\endcsname{\let\PY@bf=\textbf\def\PY@tc##1{\textcolor[rgb]{0.00,0.00,0.50}{##1}}}

\def\PYZbs{\char`\\}
\def\PYZus{\char`\_}
\def\PYZob{\char`\{}
\def\PYZcb{\char`\}}
\def\PYZca{\char`\^}
\def\PYZam{\char`\&}
\def\PYZlt{\char`\<}
\def\PYZgt{\char`\>}
\def\PYZsh{\char`\#}
\def\PYZpc{\char`\%}
\def\PYZdl{\char`\$}
\def\PYZhy{\char`\-}
\def\PYZsq{\char`\'}
\def\PYZdq{\char`\"}
\def\PYZti{\char`\~}
% for compatibility with earlier versions
\def\PYZat{@}
\def\PYZlb{[}
\def\PYZrb{]}
\makeatother


    % Exact colors from NB
    \definecolor{incolor}{rgb}{0.0, 0.0, 0.5}
    \definecolor{outcolor}{rgb}{0.545, 0.0, 0.0}



    
    % Prevent overflowing lines due to hard-to-break entities
    \sloppy 
    % Setup hyperref package
    \hypersetup{
      breaklinks=true,  % so long urls are correctly broken across lines
      colorlinks=true,
      urlcolor=urlcolor,
      linkcolor=linkcolor,
      citecolor=citecolor,
      }
    % Slightly bigger margins than the latex defaults
    
    \geometry{verbose,tmargin=1in,bmargin=1in,lmargin=1in,rmargin=1in}
    
    

    \begin{document}
    
    
    \maketitle
    
    

    
    \section{Lecture 4}\label{lecture-4}

    More examples please: on the board More questions: 4

logical equivalences: converse, contrapositive, inverse etc. - show
equivalences

other equivalences list examples use, fasteer than truthtable example de
morgan's

Analogies with set theory propositional functions sets, truth sets

Reading on sets: Rosen 2.1 and 2.2

Homework, HTPI from last year + functionally complete + more?

    \begin{Verbatim}[commandchars=\\\{\}]
{\color{incolor}In [{\color{incolor}11}]:} \PY{o}{\PYZpc{}}\PY{k}{matplotlib} inline
         \PY{o}{\PYZpc{}}\PY{k}{config} InlineBackend.figure\PYZus{}format=\PYZsq{}retina\PYZsq{}
         \PY{c+c1}{\PYZsh{} import libraries}
         \PY{k+kn}{import} \PY{n+nn}{numpy} \PY{k}{as} \PY{n+nn}{np}
         \PY{k+kn}{import} \PY{n+nn}{matplotlib} \PY{k}{as} \PY{n+nn}{mp}
         \PY{k+kn}{import} \PY{n+nn}{pandas} \PY{k}{as} \PY{n+nn}{pd}
         \PY{k+kn}{import} \PY{n+nn}{matplotlib}\PY{n+nn}{.}\PY{n+nn}{pyplot} \PY{k}{as} \PY{n+nn}{plt}
         \PY{k+kn}{import} \PY{n+nn}{laUtilities} \PY{k}{as} \PY{n+nn}{ut}
         \PY{k+kn}{import} \PY{n+nn}{slideUtilities} \PY{k}{as} \PY{n+nn}{sl}
         \PY{k+kn}{import} \PY{n+nn}{demoUtilities} \PY{k}{as} \PY{n+nn}{dm}
         \PY{k+kn}{import} \PY{n+nn}{prettytable}
         \PY{k+kn}{from} \PY{n+nn}{IPython}\PY{n+nn}{.}\PY{n+nn}{display} \PY{k}{import} \PY{n}{Image}
         \PY{k+kn}{from} \PY{n+nn}{IPython}\PY{n+nn}{.}\PY{n+nn}{display} \PY{k}{import} \PY{n}{display\PYZus{}html}
         \PY{k+kn}{from} \PY{n+nn}{IPython}\PY{n+nn}{.}\PY{n+nn}{display} \PY{k}{import} \PY{n}{display}
         \PY{k+kn}{from} \PY{n+nn}{IPython}\PY{n+nn}{.}\PY{n+nn}{display} \PY{k}{import} \PY{n}{Math}
         \PY{k+kn}{from} \PY{n+nn}{IPython}\PY{n+nn}{.}\PY{n+nn}{display} \PY{k}{import} \PY{n}{Latex}
         \PY{k+kn}{from} \PY{n+nn}{truths} \PY{k}{import} \PY{n}{Truths}
         \PY{c+c1}{\PYZsh{}reload(dm)}
         \PY{c+c1}{\PYZsh{}reload(ut)}
         \PY{c+c1}{\PYZsh{}print(\PYZsq{}\PYZsq{})}
\end{Verbatim}

    \begin{Verbatim}[commandchars=\\\{\}]
{\color{incolor}In [{\color{incolor}12}]:} \PY{o}{\PYZpc{}\PYZpc{}}\PY{k}{html}
         \PYZlt{}style\PYZgt{}
          .container.slides .celltoolbar, .container.slides .hide\PYZhy{}in\PYZhy{}slideshow \PYZob{}
             display: None ! important;
         \PYZcb{}
         \PYZlt{}/style\PYZgt{}
\end{Verbatim}

    
    \begin{verbatim}
<IPython.core.display.HTML object>
    \end{verbatim}

    
    \begin{Verbatim}[commandchars=\\\{\}]
{\color{incolor}In [{\color{incolor} }]:} 
\end{Verbatim}

    \subsection{Logical Equivalences}\label{logical-equivalences}

\begin{itemize}
\tightlist
\item
  Useful laws to simplify and evaluate propositions:

  \begin{itemize}
  \tightlist
  \item
    Examples for Disjunction and Conjunctions:

    \begin{itemize}
    \tightlist
    \item
      Associative Laws
    \item
      Distribute Laws
    \item
      Commutative Laws
    \end{itemize}
  \end{itemize}
\end{itemize}

\textbf{Tip 1}: You do not need to remember the names of the other laws.
\textbf{Tip 2:} We write \(P \equiv Q\) to say that \(P\) and \(Q\) are
equivalent propositions. Note that this is slightly different from
\(P \iff Q\).

    The statement \(P \equiv Q\) tells us that \(P \iff Q\) is a tautology,
e.g., \(P\) and \(Q\) always take the same truth values. However: -
\(P \iff Q\) is a valid proposition in our propositional logic -
\(P \equiv Q\) is not a proposition. It is something we do outside of
the logic system.

    \subsubsection{Today:}\label{today}

\begin{enumerate}
\def\labelenumi{\arabic{enumi}.}
\tightlist
\item
  More equivalences:

  \begin{itemize}
  \tightlist
  \item
    involving \(\lor, \land\) and \(\neg\): De Morgan's Laws
  \item
    involving conditional statements
  \end{itemize}
\item
  Predicates, truth sets:

  \begin{itemize}
  \tightlist
  \item
    set operations
  \item
    analogy between set theory and propositional logic
  \end{itemize}
\end{enumerate}

    \subsection{Practice Question}\label{practice-question}

\textbf{Question}: Which of the following propositions is logically
equivalent to \[
\neg(P \land(Q \lor \neg P)?
\]

\begin{enumerate}
\def\labelenumi{\alph{enumi}.}
\tightlist
\item
  \$ P \lor Q;\$\\
\item
  \(\neg P \lor Q;\)\\
\item
  \(P \lor \neg Q;\)\\
\item
  \(\neg P \lor \neg Q;\)
\end{enumerate}

    Answer: d.

Truth table. How to do it with Laws?

    \subsection{De Morgan's Laws}\label{de-morgans-laws}

\textbf{Example:}

\begin{verbatim}
Both the Yankees and the Red Sox lost last night.  
\end{verbatim}

Its negation is:

\begin{verbatim}
Either the Yankees or the Red Sox won last night.
\end{verbatim}

    \subsection{De Morgan's Laws}\label{de-morgans-laws}

\begin{itemize}
\tightlist
\item
  1st Law: Negation of Conjunction is Disjunction of Negations
  \[\neg (P \land Q) \equiv \neg P \lor \neg Q.\]
\item
  2nd Law: Negation of Disjunction is Conjunction of Negations
  \[\neg (P \lor Q) \equiv \neg P \land \neg Q.\]
\end{itemize}

    \textbf{Tip}: You don't need to remember which one is 1st or 2nd.

    \subsection{Verifying De Morgan's Laws by Truth
Table}\label{verifying-de-morgans-laws-by-truth-table}

\textbf{Idea}: The truth table of the \(\land\) operator is the negation
of the truth table of the \(\lor\) operator.

    \subsection{Equivalences involving conditional
statements}\label{equivalences-involving-conditional-statements}

This is a \emph{conditional} statement: \(P \implies Q\)\\
this is its \emph{converse}: \(Q \implies P\)\\
this is its \emph{inverse}: \(\neg P\implies \neg Q\)\\
this is its \emph{contrapositive}: \(\neg Q \implies \neg P\)

    \subsection{Practice Question}\label{practice-question}

Which one of the following statements is logically equivalent to\\
\textgreater{} If at least 10 people are there, then the lecture will be
given. ?

\begin{enumerate}
\def\labelenumi{\alph{enumi}.}
\tightlist
\item
  If there are fewer than 10 people, then the lecture will not be
  given.\\
\item
  If the lecture is not given, it means that there were fewer than 10
  people there.\\
\item
  If the lecture is given, at least 10 people were there.\\
\item
  The lecture will be given if and only if at least 10 people are there.
\end{enumerate}

    \begin{enumerate}
\def\labelenumi{\alph{enumi}.}
\setcounter{enumi}{1}
\tightlist
\item
  Contrapositive
\end{enumerate}

    \subsection{Equivalence of contrapositive via
laws}\label{equivalence-of-contrapositive-via-laws}

In this case of chain of equivalences is more elegant than a truth
table.

    \begin{itemize}
\tightlist
\item
  How about converse and inverse?
\end{itemize}

    Show scheme.

    \subsection{Practice Question}\label{practice-question}

For which of the following statements is the converse true? {[}Multiple
Answers!{]}

\begin{enumerate}
\def\labelenumi{\alph{enumi}.}
\tightlist
\item
  If \(x > 3\) and \(y < 2,\) then \(x > y.\)\\
\item
  If \(x > y\) then \(x - y > 0.\)\\
\item
  If \(x^2 = y^2,\) then \(x = y.\)\\
\item
  If \(x > y,\) then \(x = y.\)\\
\item
  If \(x \geq y,\) then \(x=y.\)
\end{enumerate}

    \subsubsection{We are done with logical
equivalences.}\label{we-are-done-with-logical-equivalences.}

Going forward, remember the different ways to show logical equivalence:

\begin{enumerate}
\def\labelenumi{\alph{enumi}.}
\tightlist
\item
  truth table\\
\item
  by applying laws (i.e., known logical equivalence)\\
\item
  combining both
\end{enumerate}

    \subsection{Beyond propositional
variables}\label{beyond-propositional-variables}

The propositional logic we covered so far is not very interesting: - we
are never looking inside the propositional variables, - does not convey
a mathematical meaning.

Next, we'll introduce some mathematical structure. For this purpose, I
am assuming you are familiar with basic set notation: - \(x \in A\)
means \(x\) is element of set \(A\), i.e., \(x\) belongs to A. -
\(A \subseteq B\) means that \(A\) is a subset of \(B.\) -
\(A = \{ x \in \mathbb{R} | x > 0\}\) defines \(A\) as the set of
positive real numbers.

    \subsection{Predicates}\label{predicates}

To use propositiona logic to talk about mathematical structures, we
start by introducing \textbf{predicates}:

\emph{Definition}: Predicates are propositions that contain variables.

\textbf{Examples}: - P(x): The integer x is prime. - Q(z): \$ z
\geq 0.\$

    The truth value of these propositions depends on the setting of the
variables: - P(10) is false, but P(11) is true. - Q(1) is true, but
Q(-1) is false.

    \textbf{Idea}: As the notation suggests, you should think of \(P(x)\) as
a function of \(x\):

\(P\) takes as input an element \(x\) from a \emph{universe} \(U\) and
returns a proposition \(P(x).\)

    \textbf{Note}: Be careful that you understand which universe \(P(x)\)
operates over.

\textbf{Examples}: - P(x): The integer x is prime. Universe
\(U = \mathbb{Z}\)

\begin{itemize}
\tightlist
\item
  Q(z): \$ z \geq 0.\$ Universe \(U = \mathbb{{R}\)
\end{itemize}

    TRUTH SETS SETS OPERATIONS QUANTIFIERS

    \begin{Verbatim}[commandchars=\\\{\}]
{\color{incolor}In [{\color{incolor}13}]:} \PY{c+c1}{\PYZsh{} image credit: Kenneth Rosen \PYZdq{}Discrete Mathematics and Its Applications 7th ed\PYZdq{}}
         \PY{n}{sl}\PY{o}{.}\PY{n}{hide\PYZus{}code\PYZus{}in\PYZus{}slideshow}\PY{p}{(}\PY{p}{)}
         \PY{n}{display}\PY{p}{(}\PY{n}{Image}\PY{p}{(}\PY{l+s+s2}{\PYZdq{}}\PY{l+s+s2}{images/L02/circuit.png}\PY{l+s+s2}{\PYZdq{}}\PY{p}{,} \PY{n}{width}\PY{o}{=}\PY{l+m+mi}{350}\PY{p}{)}\PY{p}{)}
\end{Verbatim}

    
    
    \begin{center}
    \adjustimage{max size={0.9\linewidth}{0.9\paperheight}}{L04LogicalEquivalences2_files/L04LogicalEquivalences2_28_1.png}
    \end{center}
    { \hspace*{\fill} \\}
    
    \begin{Verbatim}[commandchars=\\\{\}]
{\color{incolor}In [{\color{incolor}14}]:} \PY{n}{a} \PY{o}{=} \PY{n}{Truths}\PY{p}{(}\PY{p}{[}\PY{l+s+s1}{\PYZsq{}}\PY{l+s+s1}{P}\PY{l+s+s1}{\PYZsq{}}\PY{p}{,} \PY{l+s+s1}{\PYZsq{}}\PY{l+s+s1}{Q}\PY{l+s+s1}{\PYZsq{}}\PY{p}{]}\PY{p}{,} \PY{p}{[}\PY{p}{,}\PY{l+s+s1}{\PYZsq{}}\PY{l+s+s1}{P or Q}\PY{l+s+s1}{\PYZsq{}}\PY{p}{,} \PY{l+s+s1}{\PYZsq{}}\PY{l+s+s1}{R or Q}\PY{l+s+s1}{\PYZsq{}}\PY{p}{]}\PY{p}{)}\PY{o}{.}\PY{n}{\PYZus{}\PYZus{}str\PYZus{}\PYZus{}}\PY{p}{(}\PY{p}{)}\PY{p}{;}
         \PY{n}{a} \PY{o}{=} \PY{n}{a}\PY{o}{.}\PY{n}{replace}\PY{p}{(}\PY{l+s+s1}{\PYZsq{}}\PY{l+s+s1}{+}\PY{l+s+s1}{\PYZsq{}}\PY{p}{,}\PY{l+s+s1}{\PYZsq{}}\PY{l+s+s1}{|}\PY{l+s+s1}{\PYZsq{}}\PY{p}{)}\PY{p}{;}
         \PY{n}{a} \PY{o}{=} \PY{n}{a}\PY{o}{.}\PY{n}{replace}\PY{p}{(}\PY{l+s+s1}{\PYZsq{}}\PY{l+s+s1}{1}\PY{l+s+s1}{\PYZsq{}}\PY{p}{,}\PY{l+s+s1}{\PYZsq{}}\PY{l+s+s1}{T}\PY{l+s+s1}{\PYZsq{}}\PY{p}{)}\PY{p}{;}
         \PY{n}{a} \PY{o}{=} \PY{n}{a}\PY{o}{.}\PY{n}{replace}\PY{p}{(}\PY{l+s+s1}{\PYZsq{}}\PY{l+s+s1}{0}\PY{l+s+s1}{\PYZsq{}}\PY{p}{,}\PY{l+s+s1}{\PYZsq{}}\PY{l+s+s1}{F}\PY{l+s+s1}{\PYZsq{}}\PY{p}{)}\PY{p}{;}
         \PY{n+nb}{print}\PY{p}{(}\PY{n}{a}\PY{p}{)}\PY{p}{;}
\end{Verbatim}

    \begin{Verbatim}[commandchars=\\\{\}]

          File "<ipython-input-14-a7f75f3a068c>", line 1
        a = Truths(['P', 'Q'], [,'P or Q', 'R or Q']).\_\_str\_\_();
                                \^{}
    SyntaxError: invalid syntax


    \end{Verbatim}

    \subsection{Mathematical Proofs}\label{mathematical-proofs}

    In mathematical proofs (and logical deduction in general), we start with
some assumptions (i.e., premises) and derive some theses (i.e.,
conclusions).

A MATHEMATICAL PROOF:\\
\textbf{ASSUMPTION}
\(P \implies P_1 \implies P_2 \implies P_3 \implies P_4 \implies C\)
\textbf{THESIS}

To complete the proof, we must verify that all the implications are
valid (i.e., true). That means, we must check that indeed: -
\(P \implies P_1\), - \(P_1 \implies P_2\), - \(P_2 \implies P_3\), -
\(P_3 \implies P_4\), \ldots{} Each of these steps may require a
mathematical proof itself.

    Where does this stop?

\begin{itemize}
\tightlist
\item
  Axioms
\end{itemize}

    \subsection{Lecture 2 Add-on: Conditional relations between two
propositions}\label{lecture-2-add-on-conditional-relations-between-two-propositions}

Given two propositions \(P\) and \(Q\), we have seen that they can be
conditionally related in different ways.

    \begin{Verbatim}[commandchars=\\\{\}]
{\color{incolor}In [{\color{incolor} }]:} 
\end{Verbatim}


    % Add a bibliography block to the postdoc
    
    
    
    \end{document}
