
% Default to the notebook output style

    


% Inherit from the specified cell style.




    
\documentclass[11pt]{article}

    
    
    \usepackage[T1]{fontenc}
    % Nicer default font than Computer Modern for most use cases
    \usepackage{palatino}

    % Basic figure setup, for now with no caption control since it's done
    % automatically by Pandoc (which extracts ![](path) syntax from Markdown).
    \usepackage{graphicx}
    % We will generate all images so they have a width \maxwidth. This means
    % that they will get their normal width if they fit onto the page, but
    % are scaled down if they would overflow the margins.
    \makeatletter
    \def\maxwidth{\ifdim\Gin@nat@width>\linewidth\linewidth
    \else\Gin@nat@width\fi}
    \makeatother
    \let\Oldincludegraphics\includegraphics
    % Set max figure width to be 80% of text width, for now hardcoded.
    \renewcommand{\includegraphics}[1]{\Oldincludegraphics[width=.8\maxwidth]{#1}}
    % Ensure that by default, figures have no caption (until we provide a
    % proper Figure object with a Caption API and a way to capture that
    % in the conversion process - todo).
    \usepackage{caption}
    \DeclareCaptionLabelFormat{nolabel}{}
    \captionsetup{labelformat=nolabel}

    \usepackage{adjustbox} % Used to constrain images to a maximum size 
    \usepackage{xcolor} % Allow colors to be defined
    \usepackage{enumerate} % Needed for markdown enumerations to work
    \usepackage{geometry} % Used to adjust the document margins
    \usepackage{amsmath} % Equations
    \usepackage{amssymb} % Equations
    \usepackage{textcomp} % defines textquotesingle
    % Hack from http://tex.stackexchange.com/a/47451/13684:
    \AtBeginDocument{%
        \def\PYZsq{\textquotesingle}% Upright quotes in Pygmentized code
    }
    \usepackage{upquote} % Upright quotes for verbatim code
    \usepackage{eurosym} % defines \euro
    \usepackage[mathletters]{ucs} % Extended unicode (utf-8) support
    \usepackage[utf8x]{inputenc} % Allow utf-8 characters in the tex document
    \usepackage{fancyvrb} % verbatim replacement that allows latex
    \usepackage{grffile} % extends the file name processing of package graphics 
                         % to support a larger range 
    % The hyperref package gives us a pdf with properly built
    % internal navigation ('pdf bookmarks' for the table of contents,
    % internal cross-reference links, web links for URLs, etc.)
    \usepackage{hyperref}
    \usepackage{longtable} % longtable support required by pandoc >1.10
    \usepackage{booktabs}  % table support for pandoc > 1.12.2
    \usepackage[normalem]{ulem} % ulem is needed to support strikethroughs (\sout)
                                % normalem makes italics be italics, not underlines
    

    
    
    % Colors for the hyperref package
    \definecolor{urlcolor}{rgb}{0,.145,.698}
    \definecolor{linkcolor}{rgb}{.71,0.21,0.01}
    \definecolor{citecolor}{rgb}{.12,.54,.11}

    % ANSI colors
    \definecolor{ansi-black}{HTML}{3E424D}
    \definecolor{ansi-black-intense}{HTML}{282C36}
    \definecolor{ansi-red}{HTML}{E75C58}
    \definecolor{ansi-red-intense}{HTML}{B22B31}
    \definecolor{ansi-green}{HTML}{00A250}
    \definecolor{ansi-green-intense}{HTML}{007427}
    \definecolor{ansi-yellow}{HTML}{DDB62B}
    \definecolor{ansi-yellow-intense}{HTML}{B27D12}
    \definecolor{ansi-blue}{HTML}{208FFB}
    \definecolor{ansi-blue-intense}{HTML}{0065CA}
    \definecolor{ansi-magenta}{HTML}{D160C4}
    \definecolor{ansi-magenta-intense}{HTML}{A03196}
    \definecolor{ansi-cyan}{HTML}{60C6C8}
    \definecolor{ansi-cyan-intense}{HTML}{258F8F}
    \definecolor{ansi-white}{HTML}{C5C1B4}
    \definecolor{ansi-white-intense}{HTML}{A1A6B2}

    % commands and environments needed by pandoc snippets
    % extracted from the output of `pandoc -s`
    \providecommand{\tightlist}{%
      \setlength{\itemsep}{0pt}\setlength{\parskip}{0pt}}
    \DefineVerbatimEnvironment{Highlighting}{Verbatim}{commandchars=\\\{\}}
    % Add ',fontsize=\small' for more characters per line
    \newenvironment{Shaded}{}{}
    \newcommand{\KeywordTok}[1]{\textcolor[rgb]{0.00,0.44,0.13}{\textbf{{#1}}}}
    \newcommand{\DataTypeTok}[1]{\textcolor[rgb]{0.56,0.13,0.00}{{#1}}}
    \newcommand{\DecValTok}[1]{\textcolor[rgb]{0.25,0.63,0.44}{{#1}}}
    \newcommand{\BaseNTok}[1]{\textcolor[rgb]{0.25,0.63,0.44}{{#1}}}
    \newcommand{\FloatTok}[1]{\textcolor[rgb]{0.25,0.63,0.44}{{#1}}}
    \newcommand{\CharTok}[1]{\textcolor[rgb]{0.25,0.44,0.63}{{#1}}}
    \newcommand{\StringTok}[1]{\textcolor[rgb]{0.25,0.44,0.63}{{#1}}}
    \newcommand{\CommentTok}[1]{\textcolor[rgb]{0.38,0.63,0.69}{\textit{{#1}}}}
    \newcommand{\OtherTok}[1]{\textcolor[rgb]{0.00,0.44,0.13}{{#1}}}
    \newcommand{\AlertTok}[1]{\textcolor[rgb]{1.00,0.00,0.00}{\textbf{{#1}}}}
    \newcommand{\FunctionTok}[1]{\textcolor[rgb]{0.02,0.16,0.49}{{#1}}}
    \newcommand{\RegionMarkerTok}[1]{{#1}}
    \newcommand{\ErrorTok}[1]{\textcolor[rgb]{1.00,0.00,0.00}{\textbf{{#1}}}}
    \newcommand{\NormalTok}[1]{{#1}}
    
    % Additional commands for more recent versions of Pandoc
    \newcommand{\ConstantTok}[1]{\textcolor[rgb]{0.53,0.00,0.00}{{#1}}}
    \newcommand{\SpecialCharTok}[1]{\textcolor[rgb]{0.25,0.44,0.63}{{#1}}}
    \newcommand{\VerbatimStringTok}[1]{\textcolor[rgb]{0.25,0.44,0.63}{{#1}}}
    \newcommand{\SpecialStringTok}[1]{\textcolor[rgb]{0.73,0.40,0.53}{{#1}}}
    \newcommand{\ImportTok}[1]{{#1}}
    \newcommand{\DocumentationTok}[1]{\textcolor[rgb]{0.73,0.13,0.13}{\textit{{#1}}}}
    \newcommand{\AnnotationTok}[1]{\textcolor[rgb]{0.38,0.63,0.69}{\textbf{\textit{{#1}}}}}
    \newcommand{\CommentVarTok}[1]{\textcolor[rgb]{0.38,0.63,0.69}{\textbf{\textit{{#1}}}}}
    \newcommand{\VariableTok}[1]{\textcolor[rgb]{0.10,0.09,0.49}{{#1}}}
    \newcommand{\ControlFlowTok}[1]{\textcolor[rgb]{0.00,0.44,0.13}{\textbf{{#1}}}}
    \newcommand{\OperatorTok}[1]{\textcolor[rgb]{0.40,0.40,0.40}{{#1}}}
    \newcommand{\BuiltInTok}[1]{{#1}}
    \newcommand{\ExtensionTok}[1]{{#1}}
    \newcommand{\PreprocessorTok}[1]{\textcolor[rgb]{0.74,0.48,0.00}{{#1}}}
    \newcommand{\AttributeTok}[1]{\textcolor[rgb]{0.49,0.56,0.16}{{#1}}}
    \newcommand{\InformationTok}[1]{\textcolor[rgb]{0.38,0.63,0.69}{\textbf{\textit{{#1}}}}}
    \newcommand{\WarningTok}[1]{\textcolor[rgb]{0.38,0.63,0.69}{\textbf{\textit{{#1}}}}}
    
    
    % Define a nice break command that doesn't care if a line doesn't already
    % exist.
    \def\br{\hspace*{\fill} \\* }
    % Math Jax compatability definitions
    \def\gt{>}
    \def\lt{<}
    % Document parameters
    \title{L02PropositionalLogic}
    
    
    

    % Pygments definitions
    
\makeatletter
\def\PY@reset{\let\PY@it=\relax \let\PY@bf=\relax%
    \let\PY@ul=\relax \let\PY@tc=\relax%
    \let\PY@bc=\relax \let\PY@ff=\relax}
\def\PY@tok#1{\csname PY@tok@#1\endcsname}
\def\PY@toks#1+{\ifx\relax#1\empty\else%
    \PY@tok{#1}\expandafter\PY@toks\fi}
\def\PY@do#1{\PY@bc{\PY@tc{\PY@ul{%
    \PY@it{\PY@bf{\PY@ff{#1}}}}}}}
\def\PY#1#2{\PY@reset\PY@toks#1+\relax+\PY@do{#2}}

\expandafter\def\csname PY@tok@mb\endcsname{\def\PY@tc##1{\textcolor[rgb]{0.40,0.40,0.40}{##1}}}
\expandafter\def\csname PY@tok@vc\endcsname{\def\PY@tc##1{\textcolor[rgb]{0.10,0.09,0.49}{##1}}}
\expandafter\def\csname PY@tok@nd\endcsname{\def\PY@tc##1{\textcolor[rgb]{0.67,0.13,1.00}{##1}}}
\expandafter\def\csname PY@tok@c1\endcsname{\let\PY@it=\textit\def\PY@tc##1{\textcolor[rgb]{0.25,0.50,0.50}{##1}}}
\expandafter\def\csname PY@tok@kc\endcsname{\let\PY@bf=\textbf\def\PY@tc##1{\textcolor[rgb]{0.00,0.50,0.00}{##1}}}
\expandafter\def\csname PY@tok@mh\endcsname{\def\PY@tc##1{\textcolor[rgb]{0.40,0.40,0.40}{##1}}}
\expandafter\def\csname PY@tok@gr\endcsname{\def\PY@tc##1{\textcolor[rgb]{1.00,0.00,0.00}{##1}}}
\expandafter\def\csname PY@tok@mo\endcsname{\def\PY@tc##1{\textcolor[rgb]{0.40,0.40,0.40}{##1}}}
\expandafter\def\csname PY@tok@ow\endcsname{\let\PY@bf=\textbf\def\PY@tc##1{\textcolor[rgb]{0.67,0.13,1.00}{##1}}}
\expandafter\def\csname PY@tok@sh\endcsname{\def\PY@tc##1{\textcolor[rgb]{0.73,0.13,0.13}{##1}}}
\expandafter\def\csname PY@tok@cp\endcsname{\def\PY@tc##1{\textcolor[rgb]{0.74,0.48,0.00}{##1}}}
\expandafter\def\csname PY@tok@k\endcsname{\let\PY@bf=\textbf\def\PY@tc##1{\textcolor[rgb]{0.00,0.50,0.00}{##1}}}
\expandafter\def\csname PY@tok@s1\endcsname{\def\PY@tc##1{\textcolor[rgb]{0.73,0.13,0.13}{##1}}}
\expandafter\def\csname PY@tok@sb\endcsname{\def\PY@tc##1{\textcolor[rgb]{0.73,0.13,0.13}{##1}}}
\expandafter\def\csname PY@tok@mi\endcsname{\def\PY@tc##1{\textcolor[rgb]{0.40,0.40,0.40}{##1}}}
\expandafter\def\csname PY@tok@go\endcsname{\def\PY@tc##1{\textcolor[rgb]{0.53,0.53,0.53}{##1}}}
\expandafter\def\csname PY@tok@gs\endcsname{\let\PY@bf=\textbf}
\expandafter\def\csname PY@tok@cpf\endcsname{\let\PY@it=\textit\def\PY@tc##1{\textcolor[rgb]{0.25,0.50,0.50}{##1}}}
\expandafter\def\csname PY@tok@cs\endcsname{\let\PY@it=\textit\def\PY@tc##1{\textcolor[rgb]{0.25,0.50,0.50}{##1}}}
\expandafter\def\csname PY@tok@nf\endcsname{\def\PY@tc##1{\textcolor[rgb]{0.00,0.00,1.00}{##1}}}
\expandafter\def\csname PY@tok@nl\endcsname{\def\PY@tc##1{\textcolor[rgb]{0.63,0.63,0.00}{##1}}}
\expandafter\def\csname PY@tok@nc\endcsname{\let\PY@bf=\textbf\def\PY@tc##1{\textcolor[rgb]{0.00,0.00,1.00}{##1}}}
\expandafter\def\csname PY@tok@gh\endcsname{\let\PY@bf=\textbf\def\PY@tc##1{\textcolor[rgb]{0.00,0.00,0.50}{##1}}}
\expandafter\def\csname PY@tok@kp\endcsname{\def\PY@tc##1{\textcolor[rgb]{0.00,0.50,0.00}{##1}}}
\expandafter\def\csname PY@tok@se\endcsname{\let\PY@bf=\textbf\def\PY@tc##1{\textcolor[rgb]{0.73,0.40,0.13}{##1}}}
\expandafter\def\csname PY@tok@kr\endcsname{\let\PY@bf=\textbf\def\PY@tc##1{\textcolor[rgb]{0.00,0.50,0.00}{##1}}}
\expandafter\def\csname PY@tok@ss\endcsname{\def\PY@tc##1{\textcolor[rgb]{0.10,0.09,0.49}{##1}}}
\expandafter\def\csname PY@tok@gp\endcsname{\let\PY@bf=\textbf\def\PY@tc##1{\textcolor[rgb]{0.00,0.00,0.50}{##1}}}
\expandafter\def\csname PY@tok@na\endcsname{\def\PY@tc##1{\textcolor[rgb]{0.49,0.56,0.16}{##1}}}
\expandafter\def\csname PY@tok@bp\endcsname{\def\PY@tc##1{\textcolor[rgb]{0.00,0.50,0.00}{##1}}}
\expandafter\def\csname PY@tok@gu\endcsname{\let\PY@bf=\textbf\def\PY@tc##1{\textcolor[rgb]{0.50,0.00,0.50}{##1}}}
\expandafter\def\csname PY@tok@gd\endcsname{\def\PY@tc##1{\textcolor[rgb]{0.63,0.00,0.00}{##1}}}
\expandafter\def\csname PY@tok@o\endcsname{\def\PY@tc##1{\textcolor[rgb]{0.40,0.40,0.40}{##1}}}
\expandafter\def\csname PY@tok@w\endcsname{\def\PY@tc##1{\textcolor[rgb]{0.73,0.73,0.73}{##1}}}
\expandafter\def\csname PY@tok@sr\endcsname{\def\PY@tc##1{\textcolor[rgb]{0.73,0.40,0.53}{##1}}}
\expandafter\def\csname PY@tok@ni\endcsname{\let\PY@bf=\textbf\def\PY@tc##1{\textcolor[rgb]{0.60,0.60,0.60}{##1}}}
\expandafter\def\csname PY@tok@nn\endcsname{\let\PY@bf=\textbf\def\PY@tc##1{\textcolor[rgb]{0.00,0.00,1.00}{##1}}}
\expandafter\def\csname PY@tok@err\endcsname{\def\PY@bc##1{\setlength{\fboxsep}{0pt}\fcolorbox[rgb]{1.00,0.00,0.00}{1,1,1}{\strut ##1}}}
\expandafter\def\csname PY@tok@gt\endcsname{\def\PY@tc##1{\textcolor[rgb]{0.00,0.27,0.87}{##1}}}
\expandafter\def\csname PY@tok@mf\endcsname{\def\PY@tc##1{\textcolor[rgb]{0.40,0.40,0.40}{##1}}}
\expandafter\def\csname PY@tok@si\endcsname{\let\PY@bf=\textbf\def\PY@tc##1{\textcolor[rgb]{0.73,0.40,0.53}{##1}}}
\expandafter\def\csname PY@tok@nv\endcsname{\def\PY@tc##1{\textcolor[rgb]{0.10,0.09,0.49}{##1}}}
\expandafter\def\csname PY@tok@ge\endcsname{\let\PY@it=\textit}
\expandafter\def\csname PY@tok@gi\endcsname{\def\PY@tc##1{\textcolor[rgb]{0.00,0.63,0.00}{##1}}}
\expandafter\def\csname PY@tok@sc\endcsname{\def\PY@tc##1{\textcolor[rgb]{0.73,0.13,0.13}{##1}}}
\expandafter\def\csname PY@tok@vi\endcsname{\def\PY@tc##1{\textcolor[rgb]{0.10,0.09,0.49}{##1}}}
\expandafter\def\csname PY@tok@m\endcsname{\def\PY@tc##1{\textcolor[rgb]{0.40,0.40,0.40}{##1}}}
\expandafter\def\csname PY@tok@no\endcsname{\def\PY@tc##1{\textcolor[rgb]{0.53,0.00,0.00}{##1}}}
\expandafter\def\csname PY@tok@kt\endcsname{\def\PY@tc##1{\textcolor[rgb]{0.69,0.00,0.25}{##1}}}
\expandafter\def\csname PY@tok@kn\endcsname{\let\PY@bf=\textbf\def\PY@tc##1{\textcolor[rgb]{0.00,0.50,0.00}{##1}}}
\expandafter\def\csname PY@tok@ch\endcsname{\let\PY@it=\textit\def\PY@tc##1{\textcolor[rgb]{0.25,0.50,0.50}{##1}}}
\expandafter\def\csname PY@tok@cm\endcsname{\let\PY@it=\textit\def\PY@tc##1{\textcolor[rgb]{0.25,0.50,0.50}{##1}}}
\expandafter\def\csname PY@tok@sx\endcsname{\def\PY@tc##1{\textcolor[rgb]{0.00,0.50,0.00}{##1}}}
\expandafter\def\csname PY@tok@sd\endcsname{\let\PY@it=\textit\def\PY@tc##1{\textcolor[rgb]{0.73,0.13,0.13}{##1}}}
\expandafter\def\csname PY@tok@vg\endcsname{\def\PY@tc##1{\textcolor[rgb]{0.10,0.09,0.49}{##1}}}
\expandafter\def\csname PY@tok@nb\endcsname{\def\PY@tc##1{\textcolor[rgb]{0.00,0.50,0.00}{##1}}}
\expandafter\def\csname PY@tok@c\endcsname{\let\PY@it=\textit\def\PY@tc##1{\textcolor[rgb]{0.25,0.50,0.50}{##1}}}
\expandafter\def\csname PY@tok@kd\endcsname{\let\PY@bf=\textbf\def\PY@tc##1{\textcolor[rgb]{0.00,0.50,0.00}{##1}}}
\expandafter\def\csname PY@tok@nt\endcsname{\let\PY@bf=\textbf\def\PY@tc##1{\textcolor[rgb]{0.00,0.50,0.00}{##1}}}
\expandafter\def\csname PY@tok@s\endcsname{\def\PY@tc##1{\textcolor[rgb]{0.73,0.13,0.13}{##1}}}
\expandafter\def\csname PY@tok@il\endcsname{\def\PY@tc##1{\textcolor[rgb]{0.40,0.40,0.40}{##1}}}
\expandafter\def\csname PY@tok@ne\endcsname{\let\PY@bf=\textbf\def\PY@tc##1{\textcolor[rgb]{0.82,0.25,0.23}{##1}}}
\expandafter\def\csname PY@tok@s2\endcsname{\def\PY@tc##1{\textcolor[rgb]{0.73,0.13,0.13}{##1}}}

\def\PYZbs{\char`\\}
\def\PYZus{\char`\_}
\def\PYZob{\char`\{}
\def\PYZcb{\char`\}}
\def\PYZca{\char`\^}
\def\PYZam{\char`\&}
\def\PYZlt{\char`\<}
\def\PYZgt{\char`\>}
\def\PYZsh{\char`\#}
\def\PYZpc{\char`\%}
\def\PYZdl{\char`\$}
\def\PYZhy{\char`\-}
\def\PYZsq{\char`\'}
\def\PYZdq{\char`\"}
\def\PYZti{\char`\~}
% for compatibility with earlier versions
\def\PYZat{@}
\def\PYZlb{[}
\def\PYZrb{]}
\makeatother


    % Exact colors from NB
    \definecolor{incolor}{rgb}{0.0, 0.0, 0.5}
    \definecolor{outcolor}{rgb}{0.545, 0.0, 0.0}



    
    % Prevent overflowing lines due to hard-to-break entities
    \sloppy 
    % Setup hyperref package
    \hypersetup{
      breaklinks=true,  % so long urls are correctly broken across lines
      colorlinks=true,
      urlcolor=urlcolor,
      linkcolor=linkcolor,
      citecolor=citecolor,
      }
    % Slightly bigger margins than the latex defaults
    
    \geometry{verbose,tmargin=1in,bmargin=1in,lmargin=1in,rmargin=1in}
    
    

    \begin{document}
    
    
    \maketitle
    
    

    
    \subsection{Propositional Logic}\label{propositional-logic}

    \begin{Verbatim}[commandchars=\\\{\}]
{\color{incolor}In [{\color{incolor}1}]:} \PY{o}{\PYZpc{}}\PY{k}{matplotlib} inline
        \PY{o}{\PYZpc{}}\PY{k}{config} InlineBackend.figure\PYZus{}format=\PYZsq{}retina\PYZsq{}
        \PY{c+c1}{\PYZsh{} import libraries}
        \PY{k+kn}{import} \PY{n+nn}{numpy} \PY{k}{as} \PY{n+nn}{np}
        \PY{k+kn}{import} \PY{n+nn}{matplotlib} \PY{k}{as} \PY{n+nn}{mp}
        \PY{k+kn}{import} \PY{n+nn}{pandas} \PY{k}{as} \PY{n+nn}{pd}
        \PY{k+kn}{import} \PY{n+nn}{matplotlib}\PY{n+nn}{.}\PY{n+nn}{pyplot} \PY{k}{as} \PY{n+nn}{plt}
        \PY{k+kn}{import} \PY{n+nn}{laUtilities} \PY{k}{as} \PY{n+nn}{ut}
        \PY{k+kn}{import} \PY{n+nn}{slideUtilities} \PY{k}{as} \PY{n+nn}{sl}
        \PY{k+kn}{import} \PY{n+nn}{demoUtilities} \PY{k}{as} \PY{n+nn}{dm}
        \PY{k+kn}{import} \PY{n+nn}{prettytable}
        \PY{k+kn}{from} \PY{n+nn}{IPython}\PY{n+nn}{.}\PY{n+nn}{display} \PY{k}{import} \PY{n}{Image}
        \PY{k+kn}{from} \PY{n+nn}{IPython}\PY{n+nn}{.}\PY{n+nn}{display} \PY{k}{import} \PY{n}{display\PYZus{}html}
        \PY{k+kn}{from} \PY{n+nn}{IPython}\PY{n+nn}{.}\PY{n+nn}{display} \PY{k}{import} \PY{n}{display}
        \PY{k+kn}{from} \PY{n+nn}{IPython}\PY{n+nn}{.}\PY{n+nn}{display} \PY{k}{import} \PY{n}{Math}
        \PY{k+kn}{from} \PY{n+nn}{IPython}\PY{n+nn}{.}\PY{n+nn}{display} \PY{k}{import} \PY{n}{Latex}
        \PY{k+kn}{from} \PY{n+nn}{truths} \PY{k}{import} \PY{n}{Truths}
        \PY{c+c1}{\PYZsh{}reload(dm)}
        \PY{c+c1}{\PYZsh{}reload(ut)}
        \PY{c+c1}{\PYZsh{}print(\PYZsq{}\PYZsq{})}
\end{Verbatim}

    \begin{Verbatim}[commandchars=\\\{\}]
{\color{incolor}In [{\color{incolor}2}]:} \PY{o}{\PYZpc{}\PYZpc{}}\PY{k}{html}
        \PYZlt{}style\PYZgt{}
         .container.slides .celltoolbar, .container.slides .hide\PYZhy{}in\PYZhy{}slideshow \PYZob{}
            display: None ! important;
        \PYZcb{}
        \PYZlt{}/style\PYZgt{}
\end{Verbatim}

    
    \begin{verbatim}
<IPython.core.display.HTML object>
    \end{verbatim}

    
    Inspired from the Introduction of ``How To Prove It'' by Daniel J.
Velleman. Question 1: modification of Exercise 7a) in chapter 1.1 of
HTPI

    In our first meeting, we talked briefly about puzzles. Let's start with
one today.

    \textbf{Question 1}: Jane and Pete won't both win the math prize. Pete
will win the math prize or the chemistry prize. If Jane does not win the
math prize, Pete will win the chemistry prize. Based on these
\emph{premises}, which one of the following \emph{conclusions} is true?

\begin{enumerate}
\def\labelenumi{\alph{enumi}.}
\tightlist
\item
  Neither Jane nor Pete will win the chemistry prize.\\
\item
  Pete will win the chemistry prize.\\
\item
  Jane and Pete will both win the chemistry prize.\\
\item
  Jane will win the math prize.
\end{enumerate}

    \begin{Verbatim}[commandchars=\\\{\}]
{\color{incolor}In [{\color{incolor}3}]:} \PY{o}{\PYZpc{}}\PY{o}{\PYZpc{}}\PY{o}{\PYZpc{}} \PY{n}{Correct} \PY{n}{answer} \PY{o+ow}{is} \PY{n}{b}\PY{o}{.} \PY{n}{Pete} \PY{n}{will} \PY{n}{win} \PY{n}{the} \PY{n}{chemistry} \PY{n}{prize}
            
\end{Verbatim}

    \begin{Verbatim}[commandchars=\\\{\}]
ERROR:root:Cell magic `\%\%` not found.

    \end{Verbatim}

    \textbf{Answer}: b.

\begin{itemize}
\tightlist
\item
  PREMISE 1: Jane and Pete won't both win the math prize.\\
\item
  PREMISE 2: Pete will win the math prize or the chemistry prize.
\item
  PREMISE 3: If Jane does not win the math prize, Pete will win the
  chemistry prize.
\end{itemize}

\textbf{ARGUMENT:} - If Jane does win the math prize, Peter won't by
Premise 1. -Then, By Premise 2, Pete will win the chemistry prize. -
Else, if Jane does not win the math prize, by Premise 3, Pete will win
the chemistry prize. - Hence, Pete will win the chemistry prize.

    \emph{PROPOSITIONAL LOGIC}: How do we formally reason about problems
like this?

    \subsection{Propositional Logic}\label{propositional-logic}

Notice that the context of our problem (Jane, Pete, prizes, etc.) did
not really matter to the solution. What mattered was the logical
connections between the different statements. Indeed, we could replace
each statement with a \textbf{variable}, while preserving the struture
of the problem:

\begin{itemize}
\tightlist
\item
  variable \(P\): Pete will win the math prize.
\item
  variable \(Q\): Pete will win the chemistry prize.
\item
  variable \(R\): Jane will win the math prize.
\end{itemize}

    \textbf{NOTE}: A \emph{proposition} is a declarative sentence that is
either \emph{true}(T) or \emph{false}(F), but not both. New
propositions, called \emph{compound propositions}, are formed from
existing propositions using \emph{logical operators}.

In our case, \(P,Q\) and \(R\) are \emph{propositional variables}, each
representing the corresponding proposition on the right.

    The premises of our puzzle are propositions formed by combining the
propositions \(P,Q\) and \(R\) using \emph{logical operators} (aka
logical connectives). Next, we will introduce some of the basic such
operators

    \subsection{Common Logical Operators}\label{common-logical-operators}

The most common logical operators are \emph{negation}(NOT),
\emph{disjunction}(OR), \emph{conjunction}(AND) and
\emph{implication}(IMPLIES). They are denoted as follows:

\begin{longtable}[]{@{}lll@{}}
\toprule
Symbol & Meaning & Arity\tabularnewline
\midrule
\endhead
\(\neg\) & NOT & unary\tabularnewline
\(\lor\) & OR & binary\tabularnewline
\(\land\) & AND & binary\tabularnewline
\bottomrule
\end{longtable}

    \subsection{Negation (NOT)}\label{negation-not}

\textbf{Notation:} Given a proposition \(P\), the negation of \(P\) is
denoted \(\neg P\).

To properly define the negation operator, we need to define the truth
value of \(\neg P\) as a function of the truth value of \(P\). For this
purpose, we use a \textbf{TRUTH TABLE}:

    \begin{Verbatim}[commandchars=\\\{\}]
{\color{incolor}In [{\color{incolor}4}]:} \PY{c+c1}{\PYZsh{} image credit: Kenneth Rosen \PYZdq{}Discrete Mathematics and Its Applications 7th ed\PYZdq{}}
        \PY{n}{sl}\PY{o}{.}\PY{n}{hide\PYZus{}code\PYZus{}in\PYZus{}slideshow}\PY{p}{(}\PY{p}{)}
        \PY{n}{display}\PY{p}{(}\PY{n}{Image}\PY{p}{(}\PY{l+s+s2}{\PYZdq{}}\PY{l+s+s2}{images/L02/truthtable\PYZhy{}negation.png}\PY{l+s+s2}{\PYZdq{}}\PY{p}{,} \PY{n}{width}\PY{o}{=}\PY{l+m+mi}{350}\PY{p}{)}\PY{p}{)}
\end{Verbatim}

    
    
    \begin{center}
    \adjustimage{max size={0.9\linewidth}{0.9\paperheight}}{L02PropositionalLogic_files/L02PropositionalLogic_14_1.png}
    \end{center}
    { \hspace*{\fill} \\}
    
    \textbf{Example}: - \(P\): Pete will win the math prize. - \(\neg P\):
Pete will not win the math prize.

    \subsection{Disjunction (OR)}\label{disjunction-or}

\textbf{Notation}: Given propositions \(P\) and \(Q\), their disjunction
is denoted \(P \lor Q\).

\textbf{Truth Table}:

    \begin{Verbatim}[commandchars=\\\{\}]
{\color{incolor}In [{\color{incolor}5}]:} \PY{c+c1}{\PYZsh{} image credit: Kenneth Rosen \PYZdq{}Discrete Mathematics and Its Applications 7th ed\PYZdq{}}
        \PY{n}{sl}\PY{o}{.}\PY{n}{hide\PYZus{}code\PYZus{}in\PYZus{}slideshow}\PY{p}{(}\PY{p}{)}
        \PY{n}{display}\PY{p}{(}\PY{n}{Image}\PY{p}{(}\PY{l+s+s2}{\PYZdq{}}\PY{l+s+s2}{images/L02/truthtable\PYZhy{}disjunction.png}\PY{l+s+s2}{\PYZdq{}}\PY{p}{,} \PY{n}{width}\PY{o}{=}\PY{l+m+mi}{350}\PY{p}{)}\PY{p}{)}
\end{Verbatim}

    
    
    \begin{center}
    \adjustimage{max size={0.9\linewidth}{0.9\paperheight}}{L02PropositionalLogic_files/L02PropositionalLogic_17_1.png}
    \end{center}
    { \hspace*{\fill} \\}
    
    \textbf{Example}:

\begin{itemize}
\item
  \(P\): Pete will win the math prize.
\item
  \(Q\): Pete will win the chemistry prize.
\item
  Pete will win the math prize or the chemistry prize: \(P \lor Q\).
\end{itemize}

    \subsection{Conjunction (AND)}\label{conjunction-and}

\textbf{Notation}: Given propositions \(P\) and \(Q\), their conjunction
is denoted \(P \land Q\).

\textbf{Truth Table}:

    \begin{Verbatim}[commandchars=\\\{\}]
{\color{incolor}In [{\color{incolor}6}]:} \PY{c+c1}{\PYZsh{} image credit: Kenneth Rosen \PYZdq{}Discrete Mathematics and Its Applications 7th ed\PYZdq{}}
        \PY{n}{sl}\PY{o}{.}\PY{n}{hide\PYZus{}code\PYZus{}in\PYZus{}slideshow}\PY{p}{(}\PY{p}{)}
        \PY{n}{display}\PY{p}{(}\PY{n}{Image}\PY{p}{(}\PY{l+s+s2}{\PYZdq{}}\PY{l+s+s2}{images/L02/truthtable\PYZhy{}conjunction.png}\PY{l+s+s2}{\PYZdq{}}\PY{p}{,} \PY{n}{width}\PY{o}{=}\PY{l+m+mi}{350}\PY{p}{)}\PY{p}{)}
\end{Verbatim}

    
    
    \begin{center}
    \adjustimage{max size={0.9\linewidth}{0.9\paperheight}}{L02PropositionalLogic_files/L02PropositionalLogic_20_1.png}
    \end{center}
    { \hspace*{\fill} \\}
    
    \textbf{Example}:

\begin{itemize}
\item
  \(P\): Pete will win the math prize.
\item
  \(R\): Jane will win the math prize.
\item
  Jane and Pete will both win the math prize: \(R \land P\).
\end{itemize}

    \subsection{Back To Our Puzzle}\label{back-to-our-puzzle}

    \textbf{VARIABLES}: \(P\): Pete will win the math prize.\\
\(Q\): Pete will win the chemistry prize.\\
\(R\): Jane will win the math prize.

PREMISE 1: Jane and Pete won't both win the math prize.
\[\qquad \neg ( R \land P)\]

PREMISE 2: Pete will win the math prize or the chemistry prize.
\[\qquad P \lor Q\]

PREMISE 3: If Jane does not win the math prize, Pete will win the
chemistry prize.\\
{How do we express this?}

    \subsection{What do you think?}\label{what-do-you-think}

\textbf{Question}: Let \(S\) and \(T\) be given propositions. We want to
build a truth table for the (English) proposition
\[\qquad \mbox{if } S \mbox{, then } T\].

Which of the following truth tables should we use?

\textbar{} Propositon \(S\) \textbar{} Proposition \(T\)
\textbar{}\textbar{} Answer a) \textbar{} Answer b) \textbar{} Answer c)
\textbar{} Answer d) \textbar{}
\textbar{}---\textbar{}---\textbar{}\textbar{}------------\textbar{}---------\textbar{}---------\textbar{}----------------\textbar{}
\textbar{} F \textbar{} F \textbar{}\textbar{} T \textbar{} F \textbar{}
F \textbar{} T \textbar{} \textbar{} F \textbar{} T \textbar{}\textbar{}
T \textbar{} T \textbar{} F \textbar{} T \textbar{} \textbar{} T
\textbar{} F \textbar{}\textbar{} F \textbar{} F \textbar{} F \textbar{}
T \textbar{} \textbar{} T \textbar{} T \textbar{}\textbar{} T \textbar{}
T \textbar{} T \textbar{} F \textbar{}

    The correct answer is a).

\begin{longtable}[]{@{}lll@{}}
\toprule
Proposition \(S\) & Proposition \(T\) & If \(S\) then
\(T\)\tabularnewline
\midrule
\endhead
F & F & T\tabularnewline
F & T & T\tabularnewline
T & F & F\tabularnewline
T & T & T\tabularnewline
\bottomrule
\end{longtable}

\textbf{Think}: What's the reason behind the first two rows?

    \subsection{Another Logical Connective:
Implication}\label{another-logical-connective-implication}

    \textbf{Notation}: Given propositions \(P\) and \(Q\), the implication
\(P\) implies \(Q\) is denoted by \(P \implies Q\).

\textbf{Truth Table}:

    \begin{Verbatim}[commandchars=\\\{\}]
{\color{incolor}In [{\color{incolor}7}]:} \PY{c+c1}{\PYZsh{} image credit: Kenneth Rosen \PYZdq{}Discrete Mathematics and Its Applications 7th ed\PYZdq{}}
        \PY{n}{sl}\PY{o}{.}\PY{n}{hide\PYZus{}code\PYZus{}in\PYZus{}slideshow}\PY{p}{(}\PY{p}{)}
        \PY{n}{display}\PY{p}{(}\PY{n}{Image}\PY{p}{(}\PY{l+s+s2}{\PYZdq{}}\PY{l+s+s2}{images/L02/truthtable\PYZhy{}implication.png}\PY{l+s+s2}{\PYZdq{}}\PY{p}{,} \PY{n}{width}\PY{o}{=}\PY{l+m+mi}{350}\PY{p}{)}\PY{p}{)}
\end{Verbatim}

    
    
    \begin{center}
    \adjustimage{max size={0.9\linewidth}{0.9\paperheight}}{L02PropositionalLogic_files/L02PropositionalLogic_28_1.png}
    \end{center}
    { \hspace*{\fill} \\}
    
    \textbf{Example}: \(R\): Jane will win the math prize.\\
\(Q\): Pete will win the chemistry prize.

If Jane does not win the math prize, Pete will win the chemistry prize.
\[ \qquad (\neg R) \implies Q\]

    \subsubsection{Looking ahead \ldots{}}\label{looking-ahead}

    \begin{Verbatim}[commandchars=\\\{\}]
{\color{incolor}In [{\color{incolor}8}]:} \PY{c+c1}{\PYZsh{} image credit: Kenneth Rosen \PYZdq{}Discrete Mathematics and Its Applications 7th ed\PYZdq{}}
        \PY{n}{sl}\PY{o}{.}\PY{n}{hide\PYZus{}code\PYZus{}in\PYZus{}slideshow}\PY{p}{(}\PY{p}{)}
        \PY{n}{display}\PY{p}{(}\PY{n}{Image}\PY{p}{(}\PY{l+s+s2}{\PYZdq{}}\PY{l+s+s2}{images/L02/truthtable\PYZhy{}implication.png}\PY{l+s+s2}{\PYZdq{}}\PY{p}{,} \PY{n}{width}\PY{o}{=}\PY{l+m+mi}{350}\PY{p}{)}\PY{p}{)}
\end{Verbatim}

    
    
    \begin{center}
    \adjustimage{max size={0.9\linewidth}{0.9\paperheight}}{L02PropositionalLogic_files/L02PropositionalLogic_31_1.png}
    \end{center}
    { \hspace*{\fill} \\}
    
    Notice that there is a different way fo getting the same truth table:

\begin{longtable}[]{@{}lll@{}}
\toprule
Proposition P & Proposition Q & (not P) or Q\tabularnewline
\midrule
\endhead
T & T & T\tabularnewline
T & F & F\tabularnewline
F & T & T\tabularnewline
F & F & T\tabularnewline
\bottomrule
\end{longtable}

We say that \[P \implies Q\] and \[(\neg P) \lor Q\] are \emph{logically
equivalent}.

    \subsection{A Formal Solution to Our
Puzzle}\label{a-formal-solution-to-our-puzzle}

    Recall our \textbf{notation}: - variable \(P\): Pete will win the math
prize. - variable \(Q\): Pete will win the chemistry prize. - variable
\(R\): Jane will win the math prize.

PUZZLE: - PREMISE 1: Jane and Pete won't both win the math prize.\\
\[\qquad \neg (R \land P)\] - PREMISE 2: Pete will win the math prize or
the chemistry prize.\\
\[\qquad P \lor Q \] - PREMISE 3: If Jane does not win the math prize,
Pete will win the chemistry prize. \[\qquad (\neg R) \implies Q\] -
CONCLUSION: Pete will win the chemistry prize. \[\qquad Q\]

    Solution on the blackboard.

    \begin{Verbatim}[commandchars=\\\{\}]
{\color{incolor}In [{\color{incolor}9}]:} \PY{n}{a} \PY{o}{=} \PY{n}{Truths}\PY{p}{(}\PY{p}{[}\PY{l+s+s1}{\PYZsq{}}\PY{l+s+s1}{P}\PY{l+s+s1}{\PYZsq{}}\PY{p}{,} \PY{l+s+s1}{\PYZsq{}}\PY{l+s+s1}{Q}\PY{l+s+s1}{\PYZsq{}}\PY{p}{,} \PY{l+s+s1}{\PYZsq{}}\PY{l+s+s1}{R}\PY{l+s+s1}{\PYZsq{}}\PY{p}{]}\PY{p}{,} \PY{p}{[}\PY{l+s+s1}{\PYZsq{}}\PY{l+s+s1}{not(R and P)}\PY{l+s+s1}{\PYZsq{}}\PY{p}{,}\PY{l+s+s1}{\PYZsq{}}\PY{l+s+s1}{P or Q}\PY{l+s+s1}{\PYZsq{}}\PY{p}{,} \PY{l+s+s1}{\PYZsq{}}\PY{l+s+s1}{R or Q}\PY{l+s+s1}{\PYZsq{}}\PY{p}{]}\PY{p}{)}\PY{o}{.}\PY{n}{\PYZus{}\PYZus{}str\PYZus{}\PYZus{}}\PY{p}{(}\PY{p}{)}\PY{p}{;}
        \PY{n+nb}{print}\PY{p}{(}\PY{n}{a}\PY{o}{.}\PY{n}{replace}\PY{p}{(}\PY{l+s+s1}{\PYZsq{}}\PY{l+s+s1}{+}\PY{l+s+s1}{\PYZsq{}}\PY{p}{,}\PY{l+s+s1}{\PYZsq{}}\PY{l+s+s1}{|}\PY{l+s+s1}{\PYZsq{}}\PY{p}{)}\PY{p}{)}\PY{p}{;}
\end{Verbatim}

    \begin{Verbatim}[commandchars=\\\{\}]
|---|---|---|--------------|--------|--------|
| P | Q | R | not(R and P) | P or Q | R or Q |
|---|---|---|--------------|--------|--------|
| 0 | 0 | 0 |      1       |   0    |   0    |
| 0 | 0 | 1 |      1       |   0    |   1    |
| 0 | 1 | 0 |      1       |   1    |   1    |
| 0 | 1 | 1 |      1       |   1    |   1    |
| 1 | 0 | 0 |      1       |   1    |   0    |
| 1 | 0 | 1 |      0       |   1    |   1    |
| 1 | 1 | 0 |      1       |   1    |   1    |
| 1 | 1 | 1 |      0       |   1    |   1    |
|---|---|---|--------------|--------|--------|

    \end{Verbatim}

    \subsection{Another important connective:
Equivalence}\label{another-important-connective-equivalence}

    \textbf{Notation}: Given propositions \(P\) and \(Q\), the equivalence
of \(P\) and \(Q\) is denoted by \(P \iff Q\).

\textbf{Truth Table}:

    \begin{Verbatim}[commandchars=\\\{\}]
{\color{incolor}In [{\color{incolor}10}]:} \PY{c+c1}{\PYZsh{} image credit: Kenneth Rosen \PYZdq{}Discrete Mathematics and Its Applications 7th ed\PYZdq{}}
         \PY{n}{sl}\PY{o}{.}\PY{n}{hide\PYZus{}code\PYZus{}in\PYZus{}slideshow}\PY{p}{(}\PY{p}{)}
         \PY{n}{display}\PY{p}{(}\PY{n}{Image}\PY{p}{(}\PY{l+s+s2}{\PYZdq{}}\PY{l+s+s2}{images/L02/truthtable\PYZhy{}equivalence.png}\PY{l+s+s2}{\PYZdq{}}\PY{p}{,} \PY{n}{width}\PY{o}{=}\PY{l+m+mi}{350}\PY{p}{)}\PY{p}{)}
\end{Verbatim}

    
    
    \begin{center}
    \adjustimage{max size={0.9\linewidth}{0.9\paperheight}}{L02PropositionalLogic_files/L02PropositionalLogic_39_1.png}
    \end{center}
    { \hspace*{\fill} \\}
    
    \textbf{Example}:\\
\(R\): Jane will win the math prize.\\
\(Q\): Pete will win the chemistry prize.

Jane will win the math prize if and only if Pete will win the chemistry
prize. \[ \qquad R \iff Q\]

    \subsection{Trick Question}\label{trick-question}

    How many T's are there in the truth table for the compound proposition:
\[
[(\neg P) \lor Q] \iff [P \implies Q] 
\]

\begin{enumerate}
\def\labelenumi{\alph{enumi}.}
\tightlist
\item
  0\\
\item
  1\\
\item
  2\\
\item
  3\\
\item
  4
\end{enumerate}


    % Add a bibliography block to the postdoc
    
    
    
    \end{document}
